\section{Determining unstable operating performance points}
\label{sec:unstableOPPs}

We want to determine which OPPs are stable, and which OPPs are unstable, i.e.
those for which the system operates normally, and those which cause the system
to crash, respectively. Such a crash typically manifests as a kernel panic.
In particular, we would like to know which specific OPP lies on the boundary
between the stable realm and the unstable realm. Henceforth, such an OPP is
referred to as a \emph{crtitical point}. We collect data on these critical
points through the following process:
\begin{enumerate}
    \item Choose a frequency from the set of possible frequencies, and set that
        as the current CPU frequency.
    \item Decrease the CPU voltage by the smallest possible amount
        (i.e. $\frac{1}{1024}$ volts).
    \item If the system crashes, record the current voltage–frequency pair as a
        critical point; else, go to step (2).
    \item Go to step (1), choosing a previously unchosen frequency if one
        exists; else, we are done collecting data.
\end{enumerate}

The test detailed above is repeated numerous times for each possible frequency
in order to obtain a reasonable sample size from which to determine an average
voltage offset which renders the system unstable. To save time in collecting
this data, we narrow down the range in which the critical point lies for a
given frequency by only testing certain OPPs, and then take the time to more
precisely find the critical point by testing all OPPs within this narrower
range of voltage offsets.

In particular, we first find the critical point to the nearest $\frac{10}{1024}$
volts, performing three (3) repetitions. We assume that the crticial point lies
in the range between the minimum and maximum of these data points, plus-or-minus
another $\frac{10}{1024}$ volts. We then find the critical point to as much
accuracy as is possible (i.e. to the nearest $\frac{1}{1024}$ volts) by testing
every possible voltage offset within this narrower range, performing ten (10)
repetitions. We plot the mean, minimum, and maximum of these 10 data points for
each possible frequency.

Ideally, this data collection would be automated, with the testbench running
through the process detailed above on boot, recording the tested OPPs to disk,
and rebooting after a system crash to repeat the process for different
frequencies as required. In practice, this is not possible, for the following
reasons:
\begin{enumerate}
    \item Any data that ought to be written to disk may not actually be flushed
        from memory to disk before the sytem encounters a kernel panic, 
        resulting in loss of these data; and
    \item The system does not always reboot after encountering a kernel panic,
        contrary to the operating system's intended function of rebooting 30
        seconds after a kernel panic. In such situations, the machine must be
        reset manually.
\end{enumerate}

As such, we collect the data by hand with the aid of some shell scripts, which
follow.


