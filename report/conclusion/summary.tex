\section{Summary and discussion}
\label{sec:summary}

In summary, we have been able to meet our goal of successfully demonstrating
that hardware-based power management mechanisms (specifically DVFS) present in
an Intel Core desktop processor (namely a 4th generation Haswell unit) that are
exposed via software can be exploited in order to affect the result of a
computation. Though our observations are infrequent, they are undeniably
present, and the ubiquity of software-exposed DVFS features in these processors
ultimately means that the proof of concept detailed in §\ref{sec:observing-fault}
should be easily portable to any Intel Core processor manufactured to date;
if a proper attack can be developed that makes use of this attack vector, then
all Intel Core processors should be vulnerable to that attack. Moreover, the
specific processor used in our experimentation is incapable of per-core DVFS as
discussed in §\ref{sec:clkscrew-review}, demonstrating for the first time that
a power management attack is viable on such hardware.

With regard to SGX, Intel's secure execution environment, this remains to be
explored proper, but it is reasonable to assume that programs exclusively
using memory in an SGX enclave are just as vulnerable as those that do not
take advantage of SGX at all. This is because the faults we have induced occur
within the CPU itself, and not in memory, resulting in logic being carried out
within the CPU being affected; the content of CPU registers themselves are
likely to be made erroneous, resulting in the observed computational errors.

We remark that the data collection carried out in §\ref{sec:unstableOPPs} and
the experimentation conducted in §\ref{sec:observing-fault} could be improved
upon in hindsight. The decision to determine critical voltages only to the
nearest 1~mV rather than the highest possible degree of accuracy, namely to the
nearest $\frac{1}{1024}$~volts, was made due to the then-assumed necessity of
using the \code{undervolt} utility discussed in §\ref{sec:undervolt} to perform
voltage scaling. Since developing the kernel module discussed in §\ref{sec:bad-sha},
it is now understood that this data collection could be done with greater
precision. With regards to determining the optimal values to produce an
observable fault, this ought to have been done in a more systematic manner,
which would have been possible given the time to develop the necessary tools.
